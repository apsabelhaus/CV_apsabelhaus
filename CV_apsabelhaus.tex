%%%%%%%%%%%%%%%%%%%%%%%%%%%%%%%%%%%%%%%
% Deedy CV/Resume
% XeLaTeX Template
% Version 1.0 (5/5/2014)
%
% This template has been downloaded from:
% http://www.LaTeXTemplates.com
%
% Original author:
% Debarghya Das (http://www.debarghyadas.com)
% With extensive modifications by:
% Vel (vel@latextemplates.com)
% And more modifications by:
% Drew Sabelhaus (apsabelhaus@cmu.edu)
%
% License:
% CC BY-NC-SA 3.0 (http://creativecommons.org/licenses/by-nc-sa/3.0/)
%
% Important notes:
% This template needs to be compiled with XeLaTeX.
%
%%%%%%%%%%%%%%%%%%%%%%%%%%%%%%%%%%%%%%

\documentclass[letterpaper]{deedy-resume} % Use US Letter paper, change to a4paper for A4

% Some other packages:
\usepackage{etaremune} % enumerate in reverse. That's silly. Used for publications.

% Credit to... (insert here)
\newcommand\doilink[1]{\href{http://dx.doi.org/#1}{#1}}
\newcommand\doi[1]{doi:\doilink{#1}}
\usepackage{soul} % for better underline, \ul

\begin{document}

%-------------------------------------------------------------------------
%	TITLE SECTION
%------------------------------------------------------------------------

\lastupdated % Print the Last Updated text at the top right

\raggedright{ % left-align the heading
\namesection{Andrew P.}{Sabelhaus}{ % Your name
5827 Holden St. \#3, Pittsburgh PA 15232 \\
\urlstyle{same}\url{www.apsabelhaus.com} | 
\href{mailto:apsabelhaus@cmu.edu}{apsabelhaus@cmu.edu} | (301) 807-9842 | {\it he, him, his}}
}

  
%------------------------------------------------
% Education
%------------------------------------------------

% I'm writing this out manually instead of using the nice section headers.
% TO-DO: make this into new commands to make easy.

\section{Education}
\vspace{0.2cm}

%%%%%%%%%% PhD:

% First line: degree and school
\fontsubsection{Ph.D.  Mechanical Engineering}
\hfill % pushes the following line to the right side
\fontdescript{University of California, Berkeley}

% On the next line:
\fontlocation{Dissertation title: {\it Tensegrity Spines for Quadruped Robots}}
\hfill  %, have the date flush-right under the school name
\fontlocation{August 2019}

% Then, the committee members
{\fontlocation Dissertation Committee: {\it Alice M. Agogino (Chair), Andrew Packard, Claire Tomlin, Murat Arcak}}

% spacing for the next degree
%\sectionspace
\vspace{0.2cm}

%%%%%%%%%% Master's:

% First line: degree and school
\fontsubsection{M.S.  Mechanical Engineering}
\hfill % pushes the following line to the right side
\fontdescript{University of California, Berkeley}

% On the next line:
\fontlocation{Thesis: {\it Mechanism and Sensor Design for SUPERball, a Cable-Driven Tensegrity Robot}}
\hfill  %, have the date flush-right under the school name
\fontlocation{Dec. 2014}

% Then, the committee members
{\fontlocation Thesis Committee: {\it Alice M. Agogino, Dennis Lieu}}

% adding the remainder of the thesis name
%\quad \quad \quad $\;$  \fontlocation{ {\it a Cable-Driven Tensegrity Robot} }

% spacing for the next degree
%\sectionspace
\vspace{0.2cm}

%%%%%%%%%% Bachelor's:

% First line: degree and school
\fontsubsection{B.S.  Mechanical Engineering}
\hfill % pushes the following line to the right side
\fontdescript{University of Maryland, College Park}

% On the next line:
% \fontlocation{Minor in Computer Science}
\hspace{0.7cm} \fontsubsection{Minor: Computer Science}
\hfill  %, have the expected flush-right under the school name
\fontlocation{May 2012}

% spacing for the next degree
\sectionspace

%---------------------------------------------------------------------
% Professional Experience
%---------------------------------------------------------------------

% List my NASA stuff here - technically happened contemporanously but...
\section{Professional Experience}

\vspace{0.2cm}

%% CMU

{\fontdescript{Carnegie Mellon University \hfill Postdoctoral Research Associate}}

{\location{\hspace{0.3cm} Dept. of Mechanical Engineering \hfill 2019 - Present}}
{\location{\hspace{0.3cm} Soft Machines Lab ({\it{PI: Carmel Majidi}}) \hfill Pittsburgh, PA}}

\vspace{0.4cm}

%% NASA, through NSTRF

{\fontdescript{NASA Ames Research Center \hfill Visiting Technologist}}

{\location{\hspace{0.3cm} Intelligent Systems Division \hfill 2015 - 2019}}
{\location{\hspace{0.3cm} Intelligent Robotics Group and Robust Software Engineering \hfill Moffet Field, CA}}

\vspace{0.4cm}

%% NASA and Berkeley, when I was on the NSF fellowship

{{\fontdescript{University of California, Berkeley \hfill Graduate Student Researcher}}
  
{\location{\hspace{0.3cm} Dept. of Mechanical Engineering \hfill 2012-2019}}
{\location{\hspace{0.3cm} Berkeley Emergent Space Tensegrities Lab ({\it{PI: Alice Agogino}}) \hfill Berkeley, CA}}


\sectionspace % Some whitespace after the section

%\begin{itemize}

%\item {\fontdescript{ME135/235, Design of Microprocessor-Based Mechanical Systems.}} Graduate Student Instructor (GSI, similar to teaching assistant.) Department of Mechanical Engineering, UC Berkeley, spring 2018. Evaluations: {\it Total Effectiveness of Instructor: 4.7/5.0 (Undergrad.), 4.88/5.0 (Grad.)} Above departmental average in all metrics.
  
%\end{itemize}

%-----------------------------------------------------------------------
% Funding
%------------------------------------------------------------------------

% To-Do: have I helped win any other grants?

%\section{Awards + Grants}

\section{Grants + Funding}

\vspace{0.2cm}

%\sectionspace

\begin{etaremune}

\item {{\fontdescript{Intelligence Community Postdoctoral Research Fellowship.}} Office of the Director of National Intelligence. Title: {\it Rapid Deployment of Hard-to-Control Robots with Optimality Tradeoffs.} Full funding, 2020-2022.}

\item {{\fontdescript{NASA Space Technology Research Fellowship.}} National Aeronautics and Space Administration. Title: {\it Trajectory Tracking in Nonlinear, High-Order, Underactuated Robotic Systems.} Full funding, 2015-2019.}
  
\item {{\fontdescript{CITRIS Tech for Social Good Development Grant.}} University of California Center for Information Technology Research in the Interest of Society (CITRIS). Title: {\it Laika, The Robot Transport for Disaster Relief.}  Block grant, 2018.}

\item {{\fontdescript{Markowski-Leach Foundation Scholarship Award.}} {\it Awarded to LGBTQ individuals at San Francisco Bay Area institutions who ``are likely to make a substantial contribution to society.''} 2013-2014, re-awarded 2016-2018.\\ Currently the only repeated awardee on record.}

\item {{\fontdescript{NSF Graduate Research Fellowship.}} National Science Foundation. Full funding, 2012-2015.}

\end{etaremune}

%  Featured on ABC News: \url{https://abc7news.com/5460516/}

% Adding a quick note about other work I've assisted in writing.
% This is adjusted so that it's small and indented in.
\vspace{-0.2cm}
% \hspace{1cm} {\small \it *In addition to these independent awards and grants, Drew has assisted in writing three large multiple-PI proposals (NSF), one large single-PI\\ \hspace{1cm} proposal (NASA), and two small single-PI proposals.}


%-----------------------------------------------------------------------
% Publications
%------------------------------------------------------------------------

%\vspace{0.2cm}
%\section{Selected Publications}
%\vspace{0.2cm}
%
%\begin{etaremune}
%
%\item {\fontdescript{Model-Predictive Control of a Flexible Spine Robot.}} \underline{Sabelhaus, A.P.}; Akella, A.K.; Ahmad, Z.A.; SunSpiral, V.; {\it American Control Conference (ACC),} May 2017.
%
%\item {\fontdescript{System Design and Locomotion of SUPERball, an Untethered Tensegrity Robot.}} \underline{Sabelhaus, A.P.}; Bruce, J.; Caluwaerts, K.; Manovi, P.; Fallah Firoozi, R.; Dobi, S.; Agogino, A.M.; SunSpiral, V.; {\it IEEE International Conference on Robotics and Automation (ICRA),} May 2015.
%
%\end{etaremune}


\vspace{0.2cm}
\section{Peer-Reviewed Publications}
\vspace{0.2cm}

% For the publication titles, I use the 'description' font from the class.
%\vspace{0.4cm}

%\begin{tightitemize}
%\begin{itemize}
\begin{etaremune}

\item Z. Patterson, \underline{{\fontdescript{A.P. Sabelhaus}}}, K. Chin, C. Majidi, ``An Untethered Brittle Star Robot for Closed-Loop Underwater Locomotion.'' {\it IEEE/RSJ International Conference on Intelligent Robots and Systems (IROS),} to appear, 2020.

\item \underline{{\fontdescript{A.P. Sabelhaus}}}, A.H. Li, K.A. Sover, J. Madden, A. Barkan, A.K. Agogino, A.M. Agogino, ``Inverse Statics Optimization for Compound Tensegrity Robots.'' {\it IEEE Robotics and Automation Letters}, 2020. \doi{10.1109/LRA.2020.2983699}

\item \underline{{\fontdescript{A.P. Sabelhaus}}}, H. Zhao, E. Zhu, A.K. Agogino, A.M. Agogino, ``Model-Predictive Control with Inverse Statics Optimization for Tensegrity Spine Robots.'' {\it IEEE Transactions on Control System Technology}, 2020. \doi{10.1109/TCST.2020.2975138}

\item L.H. Chen, M.C. Daly, \underline{{\fontdescript{A.P. Sabelhaus}}}, L.A. Janse van Vuuren, H.J. Garnier, M.I. Verdugo, E. Tang, C.U. Spangenberg, F. Ghahani, A.K. Agogino, A.M. Agogino, ``Modular Elastic Lattice Platform for Rapid Prototyping of Tensegrity Robots.'' {\it  ASME International Design Engineering Technical Conferences (IDETC) / 41st Mechanisms and Robotics Conference}, Aug 2017. \doi{10.1115/DETC2017-68264}

\item \underline{{\fontdescript{A.P. Sabelhaus}}}, A.K. Akella, Z.A. Ahmad, V. SunSpiral, ``Model-Predictive Control of a Flexible Spine Robot.'' {\it American Control Conference (ACC)}, IEEE, May 2017. \doi{10.23919/ACC.2017.7963738}

\item K. Zampaglione, \underline{{\fontdescript{A.P. Sabelhaus}}}, L.H. Chen, A.M. Agogino,  A.K. Agogino, ``DNA-Structured Linear Actuators.'' {\it ASME International Design Engineering Technical Conferences (IDETC) / 40th Mechanisms and Robotics Conference}, Aug 2016. \doi{10.1115/DETC2016-60291}
  
\item \underline{{\fontdescript{A.P. Sabelhaus}}}, H. Ji, P. Hylton, Y. Madaan, C. Yang, J. Friesen, V. SunSpiral, A.M. Agogino, ``Mechanism Design and Simulation of the ULTRA Spine, a Tensegrity Robot.'' {\it ASME International Design Engineering Technical Conferences (IDETC) / 39th Mechanisms and Robotics Conference}, Aug 2015. \doi{10.1115/DETC2015-47583}

\item \underline{{\fontdescript{A.P. Sabelhaus}}}, J. Bruce, K. Caluwaerts, P. Manovi, R.F. Firoozi, S. Dobi, A.M. Agogino, V. SunSpiral, ``System Design and Locomotion of SUPERball, an Untethered Tensegrity Robot.'' {\it IEEE International Conference on Robotics and Automation (ICRA),} May 2015. \doi{10.1109/ICRA.2015.7139590}

\item K. Caluwaerts, J. Despraz, A. Iscen, \underline{{\fontdescript{A.P. Sabelhaus}}}, J. Bruce, B. Schrauwen, V. SunSpiral, ``Design and Control of Compliant Tensegrity Robots through Simulation and Hardware Validation.'' {\it Journal of the Royal Society Interface,} 2014. \doi{10.1098/rsif.2014.0520}

\item \underline{{\fontdescript{A.P. Sabelhaus}}}; J. Bruce, K. Caluwaerts, Y. Chen, D. Lu, Y. Liu, A.K. Agogino, V. SunSpiral, A.M. Agogino, ``Hardware Design and Testing of SUPERball, a Modular Tensegrity Robot.'' {\it The 6th World Conference on Structural Control and Monitoring (6WCSCM),} July 2014.

\item J. Bruce, \underline{{\fontdescript{A.P. Sabelhaus}}}, Y. Chen, D.Lu, K. Morse, S. Milam, K. Caluwaerts, A.M. Agogino, V. SunSpiral, ``SUPERball: Exploring Tensegrities for Planetary Probes.'' {\it 12th International Symposium on Artificial Intelligence, Robotics, and Automation in Space (i-SAIRAS),} June 2014.

\item J. Bruce, K. Caluwaerts, A. Iscen, \underline{{\fontdescript{A.P. Sabelhaus}}}, V. SunSpiral, ``Design and Evolution of a Modular Tensegrity Robot Platform.'' {\it IEEE International Conference on Robotics and Automation (ICRA),} May 2014. \doi{10.1109/ICRA.2014.6907361}

\item \underline{{\fontdescript{A.P. Sabelhaus}}}, D. Mirsky, L.M. Hill, S. Bergbreiter, ``TinyTeRP: A Tiny Terrestrial Robotic Platform with Modular Sensing.'' {\it IEEE International Conference on Robotics and Automation (ICRA),} May 2013. \doi{ 10.1109/ICRA.2013.6630933}


%\end{tightitemize}
%\end{itemize}
\end{etaremune}



%Total number of citations: 252


%\sectionspace

%\end{minipage}% <- ensures no space at the end
%\newpage % Start a new page
%\begin{minipage}[t]{1\textwidth}% This minipage is the full width of the page.

%--------------------------------------------------------------------
% Workshop Publications (these went through peer review, but only by organizers.)
%--------------------------------------------------------------------

%%%%%%%%%%% AS OF JAN 3rd, 2020: No more workshop sections. These are all superseded by peer-reviewed publications, and I'm not always happy about the initial results in the workshops.

% \section{Workshop Publications}

% \vspace{0.2cm}

% \begin{etaremune}

% \item {\fontdescript{Inverse Kinematics for Control of Tensegrity Soft Robots: Existence and Optimality of Solutions.}} \underline{Sabelhaus, A.P.}; Agogino, A.K.; {\it IEEE/RSJ International Conference on Intelligent Robots and Systems: Workshop on Soft Robotic Modeling and Control,} Oct. 2018. Available, arXiv:1808.08252

% \item {\fontdescript{Trajectory Tracking Control of a Flexible Spine Robot, With and Without a Reference Input.}} \underline{Sabelhaus, A.P.}; Zhao, S.H.; Daly, M.C.; Tang, E.; Zhu, E.; Akella, A.K.; Ahmad, Z.A.; SunSpiral, V.; Agogino, A.M.; {\it NASA/ESA Conference on Adaptive Hardware and Systems: Structurally Adaptive Tensegrity Robots Workshop,} July 2017. Available, arXiv:1808.08309
  
% \end{etaremune}


% Do separate sections for under review (submitted, with preprints)
% and works-in-progress (no preprints)

%-----------------------------------------------------------------------
% Under review / works in progress
%------------------------------------------------------------------------

\section{Pre-prints + Publications Under Review}

\vspace{0.2cm}

\begin{etaremune}

\item \underline{{\fontdescript{A.P. Sabelhaus}}}, L.A. Janse van Vuuren, A. Joshi, E. Zhu, H.J. Garnier, K.A. Sover, J. Navarro, A.K. Agogino, V. SunSpiral, A.M. Agogino, ``Design, Simulation, and Testing of a Flexible Actuated Spine for Quadruped Robots.'' {\it Preprint Only.} Available, arXiv:1804.06527 

\end{etaremune}  

%%%%%%%%%%
  
\section{In-Preparation Publications}

\vspace{0.2cm}

\begin{etaremune}

%\item {\fontdescript{A Smooth Dynamics Model for Cable Slackness in Cable-Driven Robots.}} \underline{Sabelhaus, A.P.} {\it In preparation.}

%\item {\fontdescript{Slack Cables in Cable-Driven Robots: Modeling and Passivity-Based Control.}} \underline{Sabelhaus, A.P.}, Barkan, A.; Sover, K.; Madden, J.; Agogino, A.K.; Agogino, A.M.

\item \underline{{\fontdescript{A.P. Sabelhaus}}}, E. Tang, K. Zampaglione, L.H. Chen, A.M. Agogino, ``Double Helix Linear Actuators.''
  
%\item {\fontdescript{Quadruped Robot Spines Require Torsion for Foot-Lifting.}} \underline{Sabelhaus, A.P.}

\item \underline{{\fontdescript{A.P. Sabelhaus}}}, ``Statically-Conservative Forces in Lagrangian Systems.''

\end{etaremune}

%-----------------------------------------------------
% Patents
%-----------------------------------------------------

\section{Patents}

\vspace{0.2cm}

\begin{etaremune}

% \item {\fontdescript{DNA Structured Linear Actuator.}} Agogino, A.; Zampaglione, K.; Chen, L-H.; \underline{Sabelhaus, A.P.}; US Patent No. 10,630,208. Issued April 21, 2020.

\item A. Agogino, K. Zampaglione, L.-H. Chen, \underline{{\fontdescript{A.P. Sabelhaus}}}, ``DNA Structured Linear Actuator.'' {\it US Patent No. 10,630,208}, issued April 21, 2020.

\item L.-H. Chen, A. Agogino, M. Daly, \underline{{\fontdescript{A.P. Sabelhaus}}}, A.K. Agogino, ``Elastic Lattices for Design of Tensegrity Structures and Robots.'' {\it Under review, US Patent Application No. US20190382995A1}.

\end{etaremune}

% Removed because not very important.
%-----------------------------------------------------
% Presentations, Workshops, and Posters section.
%-----------------------------------------------------

% \section{Presentations + Posters}

% \vspace{0.1cm}

% %{\hspace{0.5cm}} {\small \it (**Asterisk indicates an invited talk or presentation.)}

% %\textasteriskcentered
% *Asterisk indicates an invited talk or invited presentation.

% %\vspace{0.1cm}

% \begin{etaremune}
  
% %\item {\fontdescript{Laika, The Four-Legged Robot with a Flexible Spine.}} Workshop on Autonomy for Future NASA Science Missions, Oct. 2018.

% \item {\Large{\textasteriskcentered}} {\fontdescript{Laika and Belka: Walking Robots with Flexible Spines.}} Workshop on Autonomy for Future NASA Science Missions, Oct. 2018. Presentation and poster, by invitation.
  
% \item {\Large{\textasteriskcentered}} {\fontdescript{Laika, The Four-Legged Robot with a Flexible Spine.}} NASA Space Tehnology Day-On-The-Hill, Nov. 2017. Poster, by invitation, presented to the United States Congress / U.S. House of Representatives.

% \item {\fontdescript{Laika, The Quadruped Robot with a Tensegrity Spine.}} Bay Area Robotics Symposium (BARS), Oct. 2017. Presentation and Poster.

% \item {\fontdescript{UC Berkeley Robotics for Disaster Relief.}} Field Innovation Team Bootcamp 5.0, March 2017. Presentation.

% \item {\fontdescript{DNA-Structured Linear Actuators.}} SKTA Innopartners IP Redux Event, Apr 2016. Presentation.

% \item {\fontdescript{ULTRA Spine Project.}} Bay Area Robotics Symposium (BARS), Oct 2015. Presentation and Poster.
  
% \item {\Large{\textasteriskcentered}} {\fontdescript{Robotics, Mechatronics, and Intelligent Systems.}} Osher Lifelong Learning Institute, Feb 2014. Invited Talk.

% \item {\fontdescript{Mechatronic Design of Tensegrity Robotic Systems for Dynamic Locomotion.}} NASA Ames Research Center Autonomous Systems Lab Poster Symposium, Aug 2013. Poster.

% \item {\fontdescript{TinyTeRP: A Tiny Terrestrial Robotic Platform.}} International Symposium on Distributed Autonomous Robotic Systems (DARS), Nov 2012. Poster.

% \end{etaremune}

%\sectionspace

%-----------------------------------------------------
% Refereed papers (when I was a reviewer)
%-----------------------------------------------------

\section{Reviewer for Journals and Conferences}

\vspace{0.2cm}

Drew has served as a reviewer for the following journals and conferences:

\vspace{0.2cm}

\begin{tightitemize}

\item IEEE Robotics and Automation Letters (RA-L), 2017-2020

\item IEEE/RSJ International Conference on Intelligent Robots and Systems (IROS), 2018, 2020

\item IEEE Transactions on Control System Technology (T-CST), 2018

\item Journal of Open-Source Software (JOSS), 2018-2019
  
\item IEEE Robotics and Automation Magazine (RA-M), 2018
  
\item IEEE International Conference on Robotics and Automation (ICRA), 2017, 2019

\item American Control Conference (ACC), 2017-2018.

\item ASME International Design Engineering Technical Conference (IDETC), 2016-2017.

\item International Journal of Space Structures, 2017.

\item IEEE Conference on Control Technology and Applications (CCTA), 2017.

\end{tightitemize}

%-----------------------------------------------------
% Teaching
%-----------------------------------------------------

\section{Teaching + Mentoring}

\vspace{0.2cm}

\begin{itemize}

\item {{\fontdescript{Mentorship of undergraduate students.}} UC Berkeley: 18 students. Carnegie Mellon University: 1 student. Diversity: 13/19 identify as minority or under-represented, 68\%.}

\item {{\fontdescript{Mentorship of graduate students.}} UC Berkeley Master of Engineering (M.Eng) program: 15 students. Diversity: 8/15 identify as minority or under-represented, 53\%.}

\item {{\fontdescript{Graduate Student Instructor (GSI).}} University of California, Berkeley
 
\location{Jan. - May, 2018 | Mech. Eng. 135/235, Design of Microprocessor-Based Mechanical Systems}
\vspace{0.1cm}
\begin{tightitemize}
\item Created course content for lab and discussion sections, delivered stand-in lectures, assisted students with projects.
\item Overall Course Evaluations: {\it Total Effectiveness of Instructor: 4.7/5.0 (Undergrad.), 4.88/5.0 (Grad.)}
\item Teaching evaluations were above department averages in every metric.
\end{tightitemize}}

\item {{\fontdescript{Outstanding Graduate Student Instructor (GSI) Award.}} University of California, Berkeley, 2019}

% \location{Awarded upon nomination of students and department.}}

\end{itemize}

%\item {{\fontdescript{Outstanding Graduate Student Instructor (GSI) Award.}} University of California, Berkeley.

%\item {{\fontdescript{Graduate Student Instructor (GSI).}} University of California, Berkeley.
  
%\location{Jan. - May, 2018 | Mech. Eng. 135/235, Design of Microprocessor-Based Mechanical Systems}
%\begin{tightitemize}
%\item Created course content for lab and discussion sections, delivered stand-in lectures, assisted students with projects.
%\item Overall Course Evaluations: {\it Total Effectiveness of Instructor: 4.7/5.0 (Undergrad.), 4.88/5.0 (Grad.)}
%\item Teaching evaluations were above department averages in every metric.
%\end{tightitemize}}




% {{\fontdescript{Outstanding Graduate Student Instructor (GSI) Award.}} University of California, Berkeley.

\vspace{0.2cm}

%---------------------------------------------------------------------
% Diversity and Outreach
%---------------------------------------------------------------------

\section{Diversity + Outreach* + Service}

%\vspace{0.4cm}
\vspace{0.2cm}

%Professional organization and departmental service work:

\begin{itemize}

\item \fontdescript{Faculty/Staff Advisor}. \fontlocation{Out in Science, Technology, Engineering, and Mathematics$^\dag$ at Carnegie Mellon University, 2020 - present.}

\item {\fontdescript{ASME Diversity and Inclusion Strategic Commitee (DISC), Advisor}.} American Society of Mechanical Engineers (ASME). Revised ASME policy P-15.11, PS16-02, and Statement on Diversity and Inclusion to include protections for transgender ASME members. June 2016 - 2018.

\item \fontdescript{ASEE LGBTQ Virtual Community of Practice, Member.} \fontlocation{American Society for Engineering Education. 2018 - present.}

\item \fontdescript{Graduate Student Search Committee, Member.} \fontlocation{UC Berkeley Mechanical Engineering Faculty Searches. 2017 - 2018.}

\item \fontdescript{Graduate Peer Advisor}. \fontlocation{UC Berkeley Mechanical Engineering - Equity, Diversity, and Inclusion Initiative. 2014 - 2015.}

\item \fontdescript{Coordinator, Chapter Leadership Programs}. \fontlocation{Out in Science, Technology, Engineering, and Mathematics$^\dag$ Incoporated. 2012 - 2013.}

\end{itemize}

% Adding a quick note about other work I've assisted in writing.
% This is adjusted so that it's small and indented in.
\vspace{-0.2cm}
% \hspace{1cm} {\small \it In addition to these formal programs, Drew has organized many lab tours and smaller outreach events, and has volunteered with programs that\\ \hspace{1cm} recruit under-represented students to UC Berkeley and Carnegie Mellon University.}

\hspace{1cm} {\small \it *Drew has organized and volunteered with many smaller events not listed here.\\ \hspace{1cm} $^\dag$Out in Science, Technology, Engineering, and Mathematics (oSTEM) is a national organization for LGBTQ science and engineering students.}

%%%%%%%%%%%%%%%%%%%%%%%%%%

%Student organization service work:

%\begin{itemize}

%\item \fontdescript{Secondary Advisor}. \fontlocation{Out in Science, Technology, Engineering, and Mathematics (oSTEM) at UC Berkeley.  Berkeley, CA, Aug 2012 - April 2014.}

%\item \fontdescript{Founder and  Facilitator}. \fontlocation{Queer Council at the University of Maryland. College Park, MD, May 2011 - April 2012.}
  
%\item \fontdescript{Chapter President}. \fontlocation{Out in Science, Technology, Engineering, and Mathematics (oSTEM) at Maryland. College Park, MD, Nov 2010 - April 2012.}

%\end{itemize}

%%%%%%%%%%%%%%%%%%%%%%%%%%

%Workshops presented:

%\begin{itemize}

%\item {\fontdescript{Facilitation Frameworks for LGBTQ College Students.}} Out in Science, Technology, Engineering, and Mathematics Incorporated (oSTEM) National Conference, Nov 2013.
  
%\item {\fontdescript{Topics in Queer Student Leadership: Assessment, Transitions, and Goal-Driven Planning.}} National Gay and Lesbian Task Force (NGLTF) Creating Change Conference, Jan 2013. Also presented at Midwest Bisexual, Lesbian, Gay, Transgender, and Allies College Conference (MBLGTACC), Feb 2013.
  
%\end{itemize}


%%%%%%%%%%%%%%%%%%%%%%%%%%

\vspace{0.2cm}
\section{Professional development}
\vspace{0.2cm}

\begin{itemize}

\item {\fontdescript{Summer Institute for Preparing Future Faculty.}} A professional development program to prepare students for academic careers. University of California, Berkeley. Completed / certified in June 2018.
  
\item {\fontdescript{Question, Persuade, Refer: Gatekeeper.}} Trained for response to mental health crises in students. University of California Berkeley Health Center, March 2018.

\item {\fontdescript{Teaching of Mechanical Engineering at the University Level.}} UC Berkeley Mechanical Engineering Department. Course on teaching pedagogy in engineering. Spring 2018.
  
\item {\fontdescript{Workshops on Teaching and Learning.}} UC Berkeley GSI Teaching and Resource Center / Academic Innovation Studio. Attended workshops on teaching pedagogy, including `How Students Learn' and `Teaming With Diversity.' Fall 2017 - Spring 2018.

\item {\fontdescript{Teaching Conference for Graduate Student Instructors.}} UC Berkeley GSI Teaching and Resource Center. Introductory pedagogy for first-time Graduate Student Instructors. Attended in Jan. 2018.

\end{itemize}


\sectionspace % Some whitespace after the section



%----------------------------------------------------------------------------------------
%	SECOND PAGE (EXAMPLE)
%----------------------------------------------------------------------------------------

%\newpage % Start a new page

%\begin{minipage}[t]{0.33\textwidth} % The left column takes up 33% of the text width of the page

%\section{Example Section}

%\end{minipage} % The end of the left column
%\hfill
%\begin{minipage}[t]{0.66\textwidth} % The right column takes up 66% of the text width of the page

%\section{Example Section 2}

%\end{minipage} % The end of the right column

%----------------------------------------------------------------------------------------

\end{document}
